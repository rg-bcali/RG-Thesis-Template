\chapter{考察}
\label{consideration}

本章では,~\ref{evaluation}の結果と関連研究のアプローチから,本研究の妥当性について考察する.

\section{関連研究}
\label{consideration:related-works}

\subsection{PlanetLab}
\label{consideration:related-works:planetlab}

惑星規模のサービスを開発するためのオープンなプラットフォームとして,PlanetLab~\cite{PlanetLab}が挙げられる.
PlanetLabは,新規サービスの開発をサポートするグローバルな研究ネットワークであり,900以上のノードから構成される.
2003年から始動し,1000人以上の研究者が分散ストレージ,ネットワークマッピング,P2Pシステム,分散ハッシュテーブルなどの
新たな技術の開発のためにPlanetLabを利用している.
それぞれのノードは仮想マシンを提供しており,ユーザに割り当てられた仮想マシンのセットはSliceと呼ばれている.
ユーザはsocket APIを通じて個別の開発環境を構築することが可能であり,sshを通して仮想マシンにアクセスしアプリケーションを
デプロイできる.

\subsection{Emulab}
\label{consideration:related-works:emulab}

分散システムや分散ネットワークを,ネットワークエミュレータによって構築された仮想的なネットワーク上で研究や開発する取り組みとしては,
Emulab~\cite{Emulab}が挙げられる.
Emulabは大規模なソフトウェアシステムであり,仮想ネットワーク内に点在するマシン同士の接続環境を自由に設定することが可能である.
ネットワークエミュレータを活用し,ローカル環境で大規模な分散システムのための開発環境を構築する手法
~\cite{RelatedWork1}も提案されている.
数台のコンピュター上に数千台の仮想環境をプロセスレベルで構築し,それらをネットワークシミュレータにより相互接続することによって,
擬似的なネットワーク環境においての動作検証を可能にするものである.

\section{本研究の妥当性}

本研究では地理的に分散したノードをある一点から一斉にコントロール可能なシステムの提案をした.
これにより,~\ref{consideration:related-works:planetlab}章で紹介したPlanetLabのような各々の仮想マシンに対してsshを通して
アクセスするシステムに比べ,より短時間かつ少ない手順ですべてのノードにアプリケーションをデプロイすることが可能であると考えられる.
本研究で提案したシステムでは,実際のネットワーク上で実験を行うことを重要視しているため,PlanetLab同様,地理的に分散したノードが存在することを
前提条件としている.
その点,~\ref{consideration:related-works:emulab}章で取り上げたネットワークシミュレータを活用する研究に比べ,ローカル
で閉じた開発環境を構築可能な点においては劣っていると考えられる.
しかし,シミュレータ以上の正確なネットワーク環境を求める場合やbsafeネットワークのようなプライベートな研究ネットワークでは,
本研究で提案したシステムが有用であると考える.
特に分散したノードが個々の研究者によって別々に管理されている場合,動作検証におけるコミュニケーションコストとヒューマンリソース
を大幅に削減することが可能である.

%%% Local Variables:
%%% mode: japanese-latex
%%% TeX-master: "../bthesis"
%%% End:
