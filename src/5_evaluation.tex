\chapter{評価}
\label{evaluation}

本章では,本研究の提案が~\ref{issue:requirements}で述べた問題解決における要件を満たしているか評価を行う。

\section{実際性}
\label{evaluation:method}

実際性の評価をするため以下二点が実現されているか確認した。

\begin{enumerate}
  \item 異なるLAN内に配置されたサーバ同士がネットワーク上で疎通できているか
  \item 複数の論理セグメントに跨ってKubernetesクラスタを構築できているか
\end{enumerate}

一点目は、あるサーバから異なるサーバに対するpingコマンドを用いて疎通性を確認した。
以下は、Worker01(192.168.20.102)からMaster01(192.168.10.101)に対してpingコマンドを使用した際の出力である。

\begin{lstlisting}[language=bash]
  $ ping 192.168.10.101
  PING 192.168.10.101 (192.168.10.101) 56(84) bytes of data.
  64 bytes from 192.168.10.101: icmp_seq=1 ttl=63 time=1.13 ms
  64 bytes from 192.168.10.101: icmp_seq=2 ttl=63 time=1.50 ms
  64 bytes from 192.168.10.101: icmp_seq=3 ttl=63 time=1.33 ms
  64 bytes from 192.168.10.101: icmp_seq=4 ttl=63 time=1.03 ms
  64 bytes from 192.168.10.101: icmp_seq=5 ttl=63 time=1.58 ms

  --- 192.168.10.101 ping statistics ---
  5 packets transmitted, 5 received, 0% packet loss, time 4006ms
  rtt min/avg/max/mdev = 1.037/1.319/1.584/0.211 ms
\end{lstlisting}

二点目は、kubectlコマンドにてクラスタを構成するノードのIPアドレスを確認し、それらが別々のセグメントに位置することを確認した。\\*

\begin{lstlisting}[language=bash]
  $ kubectl get nodes -owide
  NAME       STATUS   ROLES    AGE   VERSION   INTERNAL-IP      EXTERNAL-IP   OS-IMAGE             KERNEL-VERSION      CONTAINER-RUNTIME
  master01   Ready    master   48d   v1.16.3   192.168.10.101   <none>        Ubuntu 18.04.3 LTS   4.15.0-70-generic   docker://18.9.7
  master02   Ready    master   48d   v1.16.3   192.168.10.102   <none>        Ubuntu 18.04.3 LTS   4.15.0-70-generic   docker://18.9.7
  master03   Ready    master   48d   v1.16.3   192.168.10.103   <none>        Ubuntu 18.04.3 LTS   4.15.0-70-generic   docker://18.9.7
  worker01   Ready
  worker02   Ready
  worker03   Ready
  worker04   Ready
\end{lstlisting}

以上の結果より、論理的に隔離されたLANに跨ってKubernetesクラスタが構築可能であることを示した。
よって、地理的に分散したシステムのためのステージング環境を実際のインターネット上に構築することが可能であることが言える。

\section{統合性}
\label{evaluation:method}

以下二点を明らかにすることで、統合性の評価を行う。

\begin{enumerate}
  \item 特定のノードからステージング環境に属する全てのノードに対して一斉に指示を送ることができるか
  \item 本研究での提案手法を用いず従来の手作業を含む手法を選んだ場合、工数にどのような差が生じるか
\end{enumerate}

\section{拡張性}
\label{evaluation:method}

拡張性の評価において、以下二点におけるコストと必要時間を計測した。

\begin{enumerate}
  \item ステージング環境へのノードの追加
  \item ステージング環境へのアプリケーションの追加
\end{enumerate}

%%% Local Variables:
%%% mode: japanese-latex
%%% TeX-master: "./thesis"
%%% End:
