\chapter{評価}
\label{evaluation}
本章では,提案システムの評価について述べる.

\section{評価手法}
\label{evaluation:method}

basafaネットワークに参加する28の大学すべてで、新バージョンのbitcoindを起動させるケースにおいて、従来の工程と本研究で
提案したシステムを使用した場合の工程を比較する。

\section{評価内容}
\label{evaluation:content}

\subsection{従来の工程}

・28大学間でデプロイするbitcoindのソフトウェアとブランチを共有
・それぞれの大学で使用するノードのOSを把握
・OSに対応したデプロイの手順書を作成
・大学(その1)へコンタクト
    ・作業日程の調整
・大学(その2)へコンタクト
    ・作業日程の調整
・大学(その3)へコンタクト
    ・作業日程の調整
...(省略)
・大学(その27)へコンタクト
    ・作業日程の調整
・大学(その1)でデプロイ作業
・大学(その2)でデプロイ作業
・大学(その3)でデプロイ作業
...(省略)
・大学(その28)でデプロイ作業
・大学(その1)へコンタクト
    ・作業進捗の共有
・大学(その2)へコンタクト
    ・作業進捗の共有
・大学(その3)へコンタクト
    ・作業進捗の共有
...(省略)
・大学(その27)へコンタクト
    ・作業進捗の共有
・すべての作業で作業が完了したことを確認
```
```
・完了報告

\subsection{本研究で提案したシステムでの工程}

・Kubernetesのマニフェストファイルを更新する
・以下のコマンドでKubernetesクラスタへのコンテナのデプロイを行う
`kubectl`
・以下のコマンドで対象のコンテナが起動することを確認する
`kubectl get pods -o wide`

%%% Local Variables:
%%% mode: japanese-latex
%%% TeX-master: "./thesis"
%%% End:
