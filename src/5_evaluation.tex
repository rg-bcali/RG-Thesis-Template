\chapter{評価}
\label{evaluation}

本章では,本研究の提案が~\ref{issue:requirements}で述べた問題解決における要件を満たしているか評価を行う.

\section{実際性}
\label{evaluation:method}

実際性の評価をするため以下二点が実現されているか確認した.

\begin{enumerate}
  \item 異なるLAN内に配置されたサーバ同士がネットワーク上で疎通できているか
  \item 複数の論理セグメントに跨ってKubernetesクラスタを構築できているか
\end{enumerate}

一点目は,あるサーバから異なるサーバに対するpingコマンドを用いて疎通性を確認した.
以下は,node01(192.168.20.101)からmaster01(192.168.10.101)に対してpingコマンドを使用した際の出力である.

\begin{lstlisting}[language=bash]
  $ ping 192.168.10.101
  PING 192.168.10.101 (192.168.10.101) 56(84) bytes of data.
  64 bytes from 192.168.10.101: icmp_seq=1 ttl=63 time=1.13 ms
  64 bytes from 192.168.10.101: icmp_seq=2 ttl=63 time=1.50 ms
  64 bytes from 192.168.10.101: icmp_seq=3 ttl=63 time=1.33 ms
  64 bytes from 192.168.10.101: icmp_seq=4 ttl=63 time=1.03 ms
  64 bytes from 192.168.10.101: icmp_seq=5 ttl=63 time=1.58 ms

  --- 192.168.10.101 ping statistics ---
  5 packets transmitted, 5 received, 0% packet loss, time 4006ms
  rtt min/avg/max/mdev = 1.037/1.319/1.584/0.211 ms
\end{lstlisting}

二点目は,kubectlコマンドにてクラスタを構成するノードのIPアドレスを確認し,それらが別々のセグメントに位置することを確認した.\\*

\begin{lstlisting}[language=bash]
  $ kubectl get nodes -owide
  NAME       STATUS     ROLES    AGE     VERSION   INTERNAL-IP      EXTERNAL-IP   OS-IMAGE             KERNEL-VERSION      CONTAINER-RUNTIME
  master01   Ready      master   58d     v1.16.3   192.168.10.101   <none>        Ubuntu 18.04.3 LTS   4.15.0-70-generic   docker://18.9.7
  master02   Ready      master   58d     v1.16.3   192.168.10.102   <none>        Ubuntu 18.04.3 LTS   4.15.0-70-generic   docker://18.9.7
  master03   Ready      master   58d     v1.16.3   192.168.10.103   <none>        Ubuntu 18.04.3 LTS   4.15.0-70-generic   docker://18.9.7
  node01     Ready      <none>   8d      v1.16.3   192.168.20.101   <none>        Ubuntu 18.04.3 LTS   4.15.0-74-generic   docker://18.9.7
  node02     Ready      <none>   4d22h   v1.16.3   192.168.20.102   <none>        Ubuntu 18.04.3 LTS   4.15.0-74-generic   docker://18.9.7
  node03     Ready      <none>   6d16h   v1.16.3   192.168.30.101   <none>        Ubuntu 18.04.3 LTS   4.15.0-74-generic   docker://18.9.7
  node04     Ready      <none>   8d      v1.16.3   192.168.30.102   <none>        Ubuntu 18.04.3 LTS   4.15.0-74-generic   docker://18.9.7
\end{lstlisting}

以上の結果より,論理的に隔離されたLANに跨ってKubernetesクラスタが構築可能であることを示した.
よって,惑星規模の分散システムのためのステージング環境を実際のインターネット上に構築することが可能であることが言える.

\section{統合性}
\label{evaluation:method}

以下二点を明らかにすることで,統合性の評価を行う.

\begin{enumerate}
  \item 特定のノードからステージング環境に属する全てのノードに対して一斉に指示を送ることができるか
  \item 本研究の提案手法を用いず従来の手作業を含む手法を選んだ場合,工数にどのような差が生じるか
\end{enumerate}

一点目は,kubectlコマンドを用いてステージング環境のワーカーノードに対して同時にアプリケーションをデプロイすることができたかを確認した.\\*

\begin{lstlisting}[language=bash]
  $ kubectl create deployment --image nginx hello-world
  $ kubectl get pods -owide
  NAME                          READY   STATUS    RESTARTS   AGE     IP          NODE       NOMINATED NODE   READINESS GATES
  hello-world-c6c6778b4-5n74d   1/1     Running   0          4d22h   10.44.0.1   node01     <none>           <none>
  hello-world-c6c6778b4-6mrj4   1/1     Running   0          4d22h   10.42.0.1   node03     <none>           <none>
  hello-world-c6c6778b4-fmnxt   1/1     Running   0          4d22h   10.47.0.1   node02     <none>           <none>
  hello-world-c6c6778b4-r8b5w   1/1     Running   0          4d22h   10.44.0.2   node04   <none>           <none>
\end{lstlisting}

二点目は,まず従来の手法を用いた場合の作業工程を列挙し,提案手法との違いを定性的に評価した.
ここでは複数の大学によって構成される研究ネットワークにて,新たに開発した惑星規模の分散システムをデプロイするケースを考える.

\begin{enumerate}
  \item 各大学のリソース(OS, CPU, Memory)の共有
  \item 各大学における作業内容の確定・共有
  \item それぞれの大学のサーバ管理者とのスケジューリングの調整
  \item 各大学でのデプロイ作業
  \item 各大学からの作業完了の連絡
  \item 全大学での作業完了の共有
  \item ステージング環境の使用開始
\end{enumerate}

上に列挙したように本研究の提案手法を用いない場合,ステージング環境で何かしらの変更を行う度に開発者間での多くのコミュニケーションと手作業が生じる.
すべての作業が単独で並行に行われれば,作業中のミスによる中断やコミュニケーション不足による手戻りが発生する可能性もある.
対して本研究の提案手法では,ステージング環境が統合的に管理されており,一括ですべてのワーカーノードに対して処理を実行できる.
例えば,パッチを当てた修正版のアプリケーションを新たにデプロイする場合,作業は一コマンドで完結し.すべての処理は自動化されているため冪等性が担保される.

\section{拡張性}
\label{evaluation:method}

拡張性の評価において,新規ノード追加時の必要時間を計測した.
計測では,必要なパッケージのインストールに掛かる時間とクラスタへの参加時間の二つを対象とした.
パッケージのインストールはAnsibleを用いて自動化し,apt updateからdocker, kubeadm等のインストール,リブートまでを含んでいる.
クラスタへの参加時間は,kubeadm joinに要した時間と,マスターノードでノードの参加を確認しステータスがReadyになるまでの時間を加算したものである.

\begin{table}[htb]
  \begin{center}
    \caption{新規ノード追加時の必要時間}
    \begin{tabular}{|l|l|} \hline
      内容 & 経過時間 \\ \hline
      パッケージのインストール & 509.50s \\ \hline
      クラスタへの参加 & 105.86s \\ \hline
    \end{tabular}
  \end{center}
\end{table}

以上の結果から,新規ノードの追加に要する時間はパッケージのインストールからクラスタへの追加まで10分程度で行えることが確認できた.
短い時間でステージング環境のスケーリングを行えたことを持って,拡張性を担保できていると考えた.

%%% Local Variables:
%%% mode: japanese-latex
%%% TeX-master: "./thesis"
%%% End:
