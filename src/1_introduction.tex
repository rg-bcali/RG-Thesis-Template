\chapter{序論}
\label{introduction}

本章では本研究の背景,課題及び手法を提示し,本研究の概要を示す.

\section{地理的に分散したシステムの発達}
\label{introduction:system-growth}

本節では,地理的に分散したシステムの概要とその発達について説明する.

\subsection{地理的に分散したシステム}

本研究における地理的に分散したシステムとは,ブロックチェーンのように世界中に地理的に分散したノードがお互いに協調動作することによって
成り立つシステムを指す.
言い換えれば,地理的に分散したノードによって形成されるP2Pアプリケーションである.

従来のクライアント・サーバ方式では,システムに参加する計算機の役割が明確に分かれており,通信において常にクライアント対サーバ
で一対一の関係が成り立つ.クライアントはサーバに対してリクエストを送り,リクエストを受け取ったサーバは特定の処理に基づいて
クライアントに対しレスポンスを返す.

対してブロックチェーンのようなP2P方式のシステムでは,参加する計算機の役割は状況に応じて柔軟に変化する.
時にクライアントとして他のノードに対して要求し,時に他のノードからの要求に対して応答する場合もある.
クライアント・サーバ方式に比べて,耐障害性・冗長性・可用性において優れているのが特徴的である.

\subsection{地理的に分散したシステムの発達}

2000年代初頭,Winny~\cite{Winny}やGnutellaといったP2Pアプリケーションが頭角を現した.
それまでサービスの構成として一般的であったクライアント・サーバ方式とは異なり,それぞれのノードが対等な関係をもつ
P2Pアプリケーションに対し注目が集まったが,クライアント・サーバ方式に置き換わるまでの隆盛はなく後退していった.

しかし,2008年にsatoshi nakamotoによりBitcoinのために開発されたブロックチェーン技術が登場することによって,再度
P2Pアプリケーションへの注目が集まり,開発や研究の勢いが盛んになってきている.

\section{本研究の着目する課題と目的}
\label{introduction:issue-and-aim}

本節では,本研究で着目しているステージング環境の説明とその必要性,ならびに地理的に分散したシステムのステージング環境を
構築する際の課題点について説明し,最後に目的を明確化する.

\subsection{ステージング環境}

ステージング環境とは,本番環境での運用をする前に実際の環境を想定してシステムの動作確認を行うための環境である.
開発者が実際に開発を行うローカル環境と実運用する本番環境では,環境の差異から動作の違いが生じ,手元で正常に動作していた
ものが本番環境に反映した途端動作しなくなるといった事象が度々発生する.
そのような事態を防ぐためにローカル環境と本番環境の間に,本番環境を想定したステージング環境を構築し,本番環境へのデプロイ前に
ステージング環境にて動作を確認することで予想外の障害が発生するリスクを抑えることができる.

\subsection{ステージング環境の必要性}

システムである以上,地理的に分散したシステムにも本番環境での予想外の障害に備えたステージング環境が必要である.
加えて、地理的に分散したP2Pアプリケーションでは、インターネット上で動作させた際のシステムへの影響や、参加するノードが
増加した際の協調動作の正常性を検証する必要があると考えられる。

\subsection{ステージング環境の構築における課題点}
\label{introduction:issue-and-aim:issue}

地理的に分散したシステムにおけるステージング環境での動作確認には,各地点でのノードの設置・リソースの共有・デプロイ作業・
アップデート作業などのタスクと、それらに伴う開発者同士のコミュニケーションが必要となる。
コミュニケーション不足によるミスや人為的ミスが発生する可能性もあり,地理的に分散したシステムのステージング環境の構築は
困難であると考えられる。

\subsection{目的}
\label{introduction:issue-and-aim:aim}

本研究では,地理的に分散したシステムのためのステージング環境を世界規模のコンピュータネットワーク上に構築するための手法を
提案することを目的とする.

\section{本研究の仮説}
\label{introduction:hypothesis}

コンテナオーケストレーションツールとして開発されたKubernetesは,アプリケーションのデプロイやスケーリングを自動化し,
コンテナ化されたアプリケーションを統合管理するためのシステムである.
Kubernetesでは複数のサーバをクラスタリングすることでこれらを実現しており、クラスタを構築するためにはお互いのサーバが
IPレベルで疎通可能な状態になければならない。
そこで、地理的に分散し各々LAN内に設置されたサーバであっても、すべてのサーバがIP Reachableである要件を満たすことが出来れば、
Kubernetesクラスタの構築が理論上可能であると考えた。

本研究では,地理的に分散したノードがIP Reachableな状態を満たすためのオーバーレイネットワーク上でKubernetesクラスタ
を構築することで,~\ref{introduction:issue-and-aim:issue}で述べた課題点を解決し,~\ref{introduction:issue-and-aim:aim}で
述べた地理的に分散したシステムのためのステージング環境を構築するという目的を達成できるのではないかと考えた.

\section{本研究の手法}
\label{introduction:proposal}

本研究では,地理的に分散したノード間でOpenVPNオーバーレイネットワークを形成することでノード同士が別セグメントに
位置していたとしても相互にIP Reachableな状態にし,その上でKubernetesクラスタを構築した.
OpenVPNとはVPNネットワークを構築することができるオープンソースソフトウェアである.
異なるLANに位置するノード同士でKubernetesクラスタを構築し,コンテナ化したアプリケーションをデプロイできていることを確認し,
本システムの課題点を解決できているか推定することで要件を満たせることを確認した.

\section{本論文の構成}
\label{introduction:structure}

本論文における以降の構成は次の通りである.

~\ref{background}章では,地理的に分散したシステムとその実験的運用方法およびそれに伴う課題点について議論し,本研究の背景を明確化する.
~\ref{issue}章では,本研究で着目する問題を解決するための要件,仮説と手法について説明する.
~\ref{implementation}章では,~\ref{proposed}章で述べた提案手法について述べる.
~\ref{evaluation}章では,\ref{background}章で求められた課題に対しての評価を行い,考察する.
~\ref{consideration}章では,\ref{evaluation}章で導き出された結果と関連研究から本研究の妥当性を考察する.
~\ref{conclusion}章では,本研究のまとめと今後の課題についてまとめる.

%%% Local Variables:
%%% mode: japanese-latex
%%% TeX-master: "../thesis"
%%% End:
