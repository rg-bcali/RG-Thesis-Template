卒業論文要旨 - 2019年度 (令和元年度)
\begin{center}
\begin{large}
\begin{tabular}{|M{0.97\linewidth}|}
    \hline
      \title \\
    \hline
\end{tabular}
\end{large}
\end{center}

~ \\
本研究では、地理的分散システムの検証実験におけるコミュニケーションコストとヒューマンリソースのオーバーヘッドを解決するための
システムを提案する。
ブロックチェーンの基盤技術として知られるP2Pネットワークなどは、地理的に分散したノードによって形成されており、これらは従来の
クライアント・サーバ方式のソフトウェアに比べテストが行いづらい。
何故なら、実際に離れた地点に設置されたノードを用いてテスト用のステージング環境を構築するためには、多くの人手とその間での
コミュニケーションが必要になるからである。

そこで、OpenVPNとKubernetesを利用することで地理的かつネットワーク上で論理的に離れたノードを統合管理できるステージング環境
を構築する。特定のポイントから一斉にすべてのノードに対しての操作を行うことで、デプロイ作業やアップデート作業における
コミュニケーションコストとヒューマンリソースを解決することが可能だと考えた。

本システムの実装後、実際にネットワーク上で別セグメントに点在するノードに対して特定のポイントから一斉にソフトウェアを起動、
更新、停止できることを確認して、本研究での提案が実現可能であることを証明する。

評価として、BSafe.networkにて本システムを使用しなかった場合と使用した場合で、作業工程数にどれほどの差が生じるかを検証した。
結果、本システムを使用した場合には大きく工程数を削減することができ、本研究で課題とした地理的分散システムの検証実験における
コミュニケーションコストとヒューマンリソースのオーバーヘッドを解消可能にした。

よって本研究は、地理的分散システムの開発における検証作業の効率性を促進し、システムの堅牢性の向上に役立つと考える。
~ \\
キーワード:\\
\underline{1. 地理的分散システム},
\underline{2. ステージング環境},
\underline{3. OpenVPN},
\underline{4. Kubernetes}
\begin{flushright}
\dept \\
\author
\end{flushright}
