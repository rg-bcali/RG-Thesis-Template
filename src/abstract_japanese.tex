卒業論文要旨 - 2019年度 (令和元年度)
\begin{center}
\begin{large}
\begin{tabular}{|M{0.97\linewidth}|}
    \hline
      \title \\
    \hline
\end{tabular}
\end{large}
\end{center}

~ \\
本研究では、惑星規模の分散システムのテストを行うためのステージング環境の構築手法を提案する。
本研究における惑星規模の分散システムとは、分散システムの中でも世界中に地理的に分散したコンピュータによって構成させるシステムを指す。
例としては、ブロックチェーンがこれにあたる。
惑星規模の分散システムを支える技術としては、P2Pが挙げられる。
このようなシステムでは、参加するサーバの数や配置は常に変化し続け、開発者がシステム全体の構成を固定化することが出来ない。
よって、惑星規模の分散システムをテストする場合、ステージング環境においても地理的な場所を指定して分散配置したサーバが必要となる。
地理的に離れた場所にあるサーバを統合管理するためには、各地点のオペレータがテストやその準備に関する情報を共有した上で、アプリケーションのインストールやアップデートを手作業で行わなければならない。
テスト内容や対象のアプリケーションが更新される度に各地点での手作業を要するため、ステージング環境でのテストの準備に多くの時間を費やすことになる。

そこで,OpenVPNとKubernetesを利用することで地理的かつネットワーク上で論理的に離れたコンピュータを統合管理できるステージング環境を構築する.
特定のポイントから一斉にすべてのコンピュータに対しての操作を行うことで,デプロイ作業やアップデート作業におけるオペレータ同士のコミュニケーションや各拠点の手作業を削減できると考えた。

本システムの実装後,実際にネットワーク上で別セグメントに点在するコンピュータに対して特定のポイントから一斉にソフトウェアを起動,更新,停止できることを確認して,
本研究での提案が実現可能であることを証明する.

評価として,本システムを使用した場合と使用しなかった場合で,作業工程数にどれほどの差が生じるかを検証した.

結果,本システムを使用した場合には工程数を削減することができ,
本研究で課題とした惑星規模の分散システムのステージング環境におけるオペレータ同士のコミュニケーションや手作業のオーバーヘッドを解決できた。

よって本研究は,惑星規模の分散システムの開発におけるテストの効率性を促進し,システムの堅牢性の向上に役立つと考える.
~ \\
キーワード:\\
\underline{1. 惑星規模の分散システム},
\underline{2. ステージング環境},
\underline{3. OpenVPN},
\underline{4. Kubernetes}
\begin{flushright}
\dept \\
\author
\end{flushright}
