卒業論文要旨 - 2019年度 (令和元年度)
\begin{center}
\begin{large}
\begin{tabular}{|M{0.97\linewidth}|}
    \hline
      \title \\
    \hline
\end{tabular}
\end{large}
\end{center}

~ \\
本研究では,惑星規模の分散システムのための試験環境の構築手法を提案する.
本研究における惑星規模の分散システムとは,分散システムの中でも世界中に地理的に分散したコンピュータによって構成されるものと定義する.
惑星規模の分散システムを支える技術として,P2Pがあげられる.
P2Pシステムは,中央集権的なサーバを必要とせず,お互いに対等な関係をもつコンピュータが協調動作することによって成り立つ.
ブロックチェーンはその一例である.
惑星規模の分散システムにおけるテストでは,地理的に分散することによるネットワークでの通信の遅延を考慮する必要がある.
よって,試験環境は地理的に分散配置されたサーバで構成されるべきである.

既存の惑星規模の分散システムの試験環境としては,PlanetLabやパブリッククラウドサービスの活用,BSafe.networkがあげられるが,それぞれ課題があると考える.
PlanetLabはネットワークサービスの開発を支援する研究ネットワークであり,世界中の717地域1353のサーバに対してSSHを通して操作を行えるが,OSやCPU,メモリなどのサーバの環境を柔軟に変更できない.
パブリッククラウドサービスでは,リージョンと呼ばれるデータセンターの地域を指定することでサーバを分散配置できるが,リージョンが限定的であることに加え,公にサービスが実稼働する環境に比べ通信の遅延が少ないため環境に差が生じてしまう.
BSafe.networkは32の大学によって構成されるブロックチェーン技術の研究を行うためのネットワークであり,各大学が保有するサーバを用いて実験が行えるが,各サーバの管理権限が分かれているため複数の大学間で共同研究を行う場合はオペレータの手作業が必要である.
これらの課題を解決するため,本研究ではOpenVPNとKubernetesを組み合わせた統合的試験環境の構築手法を提案する.
本システムでは,サーバがどこに分散配置されていてもネットワークでの通信の遅延を考慮したテストが可能である.
加えて,仮想化技術を活用することでひとつのサーバ内に複数のアプリを様々な環境下で動作させることができ,各サーバが別々の管理下にある場合でも統合可能である.

本システムの実装後,すべてのサーバの統合管理が可能であるか,ならびに通信の遅延を考慮した上で本システムが正常に動作するかを検証した.

本研究は,惑星規模の分散システムのテストをより公の実稼働環境に近い形で柔軟に行うことを可能とし,BSafe.networkのように各サーバが別々の管理下にある場合でもオペレータの手作業を省いた統合的試験環境の構築が可能である.

~ \\
キーワード:\\
\underline{1. 惑星規模の分散システム},
\underline{2. 試験環境},
\underline{3. OpenVPN},
\underline{4. Kubernetes}

\begin{flushright}
\dept \\
\author
\end{flushright}
