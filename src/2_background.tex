\chapter{背景}
\label{background}

本章では本研究の背景について述べる.

\section{地理的分散システム}
本節では,地理的分散システムの具体例ついて概説する.

\subsection{Winny}
Winnyはソフトウェアエンジニア金子勇氏が開発し、2002年に発表されたファイル共有ソフトである。
システム上で中央集権的なサーバを保持せず、ノード同士が相互に接続することで実現されるP2Pアプリケーションとして注目を浴びた。
ユーザはノード内に保持されたファイルを他のノードと共有することができるため、任意のファイルをアップロードしたり、逆に他のノードが保持しているファイルをダウンロードすることができる。
Winnyでは、受信ファイルの送信元や送信ファイルの宛先をユーザが確認することはできず、バックグラウンドでの処理はユーザに見せないよう高い秘匿性が担保されていた。
従来のサーバ・クライアント方式のシステムアーキテクチャとは打って変わって出た新しい形のアプリケーションであったが、高い匿名性も起因して、
一部のユーザが違法な音楽ファイルや動画ファイル、コンピュータウイルスをWinnyにアップロードしたことで著作権法違反が問われた。
開発者である金子氏にも疑いがかけられ2004年に逮捕、その後画期的な発明であったWinnyも衰退していった。

\subsection{Gnutella}
Winnyに同じくGnutellaも中央集権型サーバに依存せず、P2Pネットワーク上のノード間の通信のみでファイルを送受信を行うファイル共有アプリケーションである。

\subsection{Bitcoin}
Bitcoinは2008年にSatoshi Nakamotoと名乗る人物によって論文にて提唱されたものである。
2009年にはソフトウェアとして実現されており、今では多くのユーザに使用されている上、仮想通貨の先駆けとして他の仮想通貨を生む大きな起点となった。
同時に、2000年代後半に勢いを失っていたP2Pシステムの存在を再度世に知らしめ、開発の促進を促す起爆剤の役割を果たしたと考えられる。
Bitcoinは基盤技術のひとつとしてWinnyやGnutellaと共通するP2Pネットワークを採用している。
参加するノードはそれぞれがシステム上のデータを保持し相互にデータを検証しあうことで、第三者的監視機関を必要とせずにデータの堅の牢性を担保することが可能である。

\section{地理的分散システムの技術}
本節では,地理的分散システムを実現するための技術について概説する.

\subsection{P2Pネットワーク}

\subsection{従来のサーバ・クライアント方式との違い}

\section{地理的分散システムとステージング環境での動作確認}
本節では、地理的分散システムのステージング環境と動作確認について概説する。

\subsection{最小限の動作確認}
最も簡単に行える動作確認は、ふたつのノード間で行うテストである。ネットワーク上の二点でそれぞれノードを立ち上げ、システムの機能が正しく動作するかを確認する。
従来のサーバ・クライアント方式では、最低限ではあるが機能の保証ができる。サーバ・クライアント方式では、中央集権的サーバとクライアントが一対一の関係で繋がっており、開発者はクライアントとの通信ただひとつに注力すればいいからだ。
ユーザが増加した場合の障害対策やレスポンスタイムの向上は確かに必要であるが、サーバとクライアントの一対一の関係性は不変であるため、ネットワーク自体が正常で有る限り問題は二点間に閉ざされておりテストがしやすい。
一方、中央集権的サーバがなくノード同士がサーバにもクライアントにもなり得る地理的分散システムでは、この方法は十分ではない。ネットワークに参加するノードが増加すれば個々のノード同士の関係性は変化し、関係性が固定されないためである。
もうひとつの理由として、ノード周辺のネットワーク環境によって動作に影響が出る可能性が考えられる。地理的分散システムの具体例として挙げたBitcoinでは、参加するノードは全て同じデータを保持する。
データの送信や受信において遅延が発生すれば何らかの影響が出ることは簡単に予想可能である。例えシステム上でデータの不整合を防ぐロジックが組まれていたとしても、ロジックを表現したコードが実際の環境で正常な動作をすることを動作確認無しで担保することは難しい。
以上の理由から、地理的分散システムの動作確認をするにあたって二点間でのステージング環境は不十分であり、より多くのノードを実際の世界規模のネットワーク上で動かしたステージング環境が必要であると考えられる。

\subsection{地理的に分散したノードによる動作確認}

\subsection{}

\subsection{}

\if0
\begin{figure}[h]
    \begin{center}
        \includegraphics[scale=0.4]{./img/hashrate.png}
        \caption{2017年1月のハッシュレート分布 出典:Blockchain.info\cite{bitcoinhashrate}}
        \label{img:hashrate}
    \end{center}
\end{figure}
\fi
