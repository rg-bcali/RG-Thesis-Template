\chapter{結論}
\label{conclusion}

本章では,本研究のまとめと今後の課題を示す.

\section{本研究のまとめ}
\label{conclusion:conclusion}

本研究では,地理的に分散したシステムの本番適応前の動作検証におけるコミュニケーションコストとヒューマンリソースのオーバーヘッドを解決するため,OpenVPNと
Kubernetesを利用したステージング環境の提案をした.着目した問題に対する解決策として実際性,統合性,拡張性の三つの要件が求められると考えた.
第一に,OpenVPNを用いることでネットワーク上で論理的に離れたノード間での疎通性を獲得した.これによって対象のノードをローカル環境から実際のインターネット上に
拡張することができ,地理的に分散したシステムの動作検証に必要である実際性を満たすことが出来たと考える.第二に拡張性については,OpenVPNによるオーバーレイネットワーク
上でKubernetesクラスタを構築することで,分散したノードに対し統合的な操作を可能にすることで解決した.第三に拡張性であるが,Kubernetesクラスタ上では
アプリケーションをコンテナ型仮想マシンとして動作させるため,容易にコンテナの追加や削除を行うことが可能である.加えて,kubeadmによりクラスタへ
新規にノードを追加することも可能であるため,ステージング環境を自由に拡張することが可能である.よって拡張性も満たしていると考えられる.

\section{本研究の課題と展望}
\label{conclusion:issue}

本研究では,OpenVPNを用いたオーバーレイネットワーク上にKubernetesクラスタを構築した.すべてのノードはVPNサーバと接続し,
Kubernetesクラスタ上での通信はすべてVPNサーバを通して行うが,本研究では参加ノードに対するVPNサーバの負荷と
Kubernetesクラスタへの影響までを測定することができなかった.実用に向けた次のステップとしては,VPNサーバの負荷とレイテンシ
についての詳細な実験をする必要性があると考えた.

%%% Local Variables:
%%% mode: japanese-latex
%%% TeX-master: "../thesis"
%%% End:
