\chapter{実装}
\label{implementation}

本章では提案手法の実装について述べる.

\section{実装環境}
\label{implementation:environment}

本節では,本研究で構築した実装環境について概説する.

\subsection{ハードウェアおよびソフトウェア}
\label{implementation:environment:resouces}

本研究で使用したハードウェアおよびソフトウェアとそのバージョンを以下に示す.

\begin{table}[htb]
  \begin{center}
    \caption{使用したハードウェアおよびソフトウェア}
    \begin{tabular}{|l|l|} \hline
      ハードウェア/ソフトウェア & 機種/バージョン \\ \hline
      Server & FUJITSU PRIMERGY S6 \\ \hline
      VMWare ESXi & 6.5 \\ \hline
      VyOS & 1.2.1 \\ \hline
      OpenVPN & 2.3.4 \\ \hline
      Ubuntu & 18.04 \\ \hline
      kubeadm & 1.16.3 \\ \hline
      kubelet & 1.16.3 \\ \hline
      kubectl & 1.16.3 \\ \hline
      HA-Proxy & 1.8.8 \\ \hline
    \end{tabular}
  \end{center}
\end{table}

\subsection{物理サーバの準備}
\label{implementation:esxi}

本研究では,実装において複数のセグメントおよびKubernetesクラスタの構築に複数のサーバが必要であったため,それらを仮想的に作成できるVMWare ESXi(以下,ESXi)を導入した.
使用したのは,ESXi6.5だ.
ESXiはホストOSを必要とせず,直接ハードウェアにインストールさせて動作させるハイパーバイザー型であるため,まず初めにESXiインストーラの
ブータブルイメージをUSBメモリに書き込み,FUJITSUサーバにインストールした.
計二台のFUJITSUサーバにESXiをインストールし,それぞれ以下のIPアドレスを設定した.

\begin{table}[htb]
  \begin{center}
    \caption{ESXiのIPアドレス}
    \begin{tabular}{|l|l|} \hline
      名前 & IPアドレス \\ \hline
      1台目 & 10.4.0.13 \\ \hline
      2台目 & 10.4.0.14 \\ \hline
    \end{tabular}
  \end{center}
\end{table}

\subsection{ネットワーク構成}
\label{implementation:network-environment}

本研究で構築したネットワーク構成について説明する.

まず初めに,ESXiの仮想スイッチとVLANを用いて二つのESXiサーバ上に新たに計三つの論理セグメントを構築した.
Vlanによって論理的にセグメントを分割することで,お互いに通信不可能な環境とした.
以下に,Vlan IDと対応するアドレスプレフィックスを示す.
なお,10.4.0.0/16のアドレスプレフィックスはVlan ID 0に対応している.

\begin{table}[htb]
  \begin{center}
    \caption{Vlan IDと対応するアドレスプレフィックス}
    \begin{tabular}{|l|l|} \hline
      Vlan ID & アドレスプレフィックス \\ \hline
      0 & 10.4.0.0/16 \\ \hline
      10 & 192.168.10.0/24 \\ \hline
      20 & 192.168.20.0/24 \\ \hline
      30 & 192.168.30.0/24 \\ \hline
    \end{tabular}
  \end{center}
\end{table}

\subsection{VMの配置}

ネットワーク構築後,Kubernetesクラスタの構築に必要なサーバをVMとして立ち上げた.
それぞれのVMのOSにはUbuntu18.04を採用した.
以下に構築したサーバの詳細を示す.

\begin{landscape}
  \begin{table}[htb]
    \begin{center}
      \caption{設置したVMの詳細}
      \begin{tabular}{|l|l|l|l|l|} \hline
        名前 & ネットワークインターフェース名 & Vlan ID & IPアドレス & 役割 \\ \hline
        lb & ens160 & 10 & 192.168.10.253 & マスターノードのロードバランサー \\ \hline
        master01 & ens160 & 10 & 192.168.10.101 & マスターノード \\ \hline
        master02 & ens160 & 10 & 192.168.10.102 & マスターノード \\ \hline
        master03 & ens160 & 10 & 192.168.10.103 & マスターノード \\ \hline
        node01 & ens160 & 20 & 192.168.20.101 & ワーカーノード \\ \hline
        node02 & ens160 & 20 & 192.168.20.102 & ワーカーノード \\ \hline
        node03 & ens160 & 30 & 192.168.30.101 & ワーカーノード \\ \hline
        node04 & ens160 & 30 & 192.168.30.102 & ワーカーノード \\ \hline
      \end{tabular}
    \end{center}
  \end{table}
\end{landscape}

\subsection{ルーターの配置}

次に各拠点にOpenVPNの設定をするルーターを設置した.
ルーターのOSにはVyOS 1.2.1,OpenVPNはバージョン2.3.4を採用した.
以下にルーターのネットワーク情報を示す.

\begin{table}[htb]
  \begin{center}
    \caption{設置したルーターの詳細}
    \begin{tabular}{|l|l|l|l|l|} \hline
      名前 & ネットワークインターフェース名 & Vlan ID & IPアドレス \\ \hline
      vyos01 & eth0 & 0 & 10.4.0.90 \\ \hline
      vyos01 & eth1 & 10 & 192.168.10.1 \\ \hline
      vyos02 & eth0 & 0 & 10.4.0.91 \\ \hline
      vyos02 & eth1 & 20 & 192.168.20.1 \\ \hline
      vyos03 & eth0 & 0 & 10.4.0.92 \\ \hline
      vyos03 & eth1 & 30 & 192.168.30.1 \\ \hline
    \end{tabular}
  \end{center}
\end{table}

全てのルーターはお互いに疎通可能である.
さらに,各拠点に設置されたサーバと疎通できるようeth1のネットワークインターフェースには別のIPアドレスを設定した.
この時点での各サーバの疎通性は以下の通りである.

\begin{table}[htb]
  \begin{center}
    \caption{OpenVPN設定前の各サーバの疎通性}
    \begin{tabular}{|c|c|c|c|c|c|c|c|c|} \hline
      & lb & master01 & master02 & master03 & node01 & node02 & node03 & node04 \\ \hline
      lb & \ & ○ & ○ & ○ & × & × & × & × \\ \hline
      master01 & ○ & \ & ○ & ○ & × & × & × & × \\ \hline
      master02 & ○ & ○ & \ & ○ & × & × & × & × \\ \hline
      master03 & ○ & ○ & ○ & \ & × & × & × & × \\ \hline
      node01 & × & × & × & × & \ & ○ & × & × \\ \hline
      node02 & × & × & × & × & ○ & \ & × & × \\ \hline
      node03 & × & × & × & × & × & × & \ & ○ \\ \hline
      node04 & × & × & × & × & × & × & ○ & \ \\ \hline
    \end{tabular}
  \end{center}
\end{table}
\label{tb:before-openvpn}

\subsection{OpenVPNの設定}

\ref{tb:before-openvpn}で示したように,OpenVPNの設定をする前ではすべてのサーバはお互いに疎通可能な状態にはない.
Kubernetesは,クラスタに参加するサーバのすべてが疎通可能,厳密にはIP reachableな環境下にある必要がある.
そこでOpenVPNを用いて,複数の分離したLANを仮想的に接続しKubernetesの要件を満たそうと試みた.
本実装では,OpenVPNのsite-to-siteモードを採用した.
client-serverモードを採用しなかった理由としては以下の二点が挙げられる.

\begin{enumerate}
  \item Kubernetesは通信時にデフォルトゲートウェイに設定したネットワークインターフェースを使用するため,サーバ毎にOpenVPNを設定するclient-serverモードではトンネルインターフェースを通して通信ができない.
  \item サーバ毎に証明書と鍵の管理が必要なため扱いづらい.
\end{enumerate}

対して,site-to-siteモードでは以下の利点が挙げられる.

\begin{enumerate}
  \item ルーティングはルーターに任せられるため,サーバは通信時にデフォルトゲートウェイに設定されたネットワークインターフェースを使用できる.
  \item OpenVPNの設定はLAN内のルーターのみ.
\end{enumerate}

以下に,OpenVPN設定後の各サーバの疎通性を示す.

\begin{table}[htb]
  \begin{center}
    \caption{OpenVPN設定前の各サーバの疎通性}
    \begin{tabular}{|c|c|c|c|c|c|c|c|c|} \hline
      & lb & master01 & master02 & master03 & node01 & node02 & node03 & node04 \\ \hline
      lb & \ & ○ & ○ & ○ & ○ & ○ & ○ & ○ \\ \hline
      master01 & ○ & \ & ○ & ○ & ○ & ○ & ○ & ○ \\ \hline
      master02 & ○ & ○ & \ & ○ & ○ & ○ & ○ & ○ \\ \hline
      master03 & ○ & ○ & ○ & \ & ○ & ○ & ○ & ○ \\ \hline
      node01 & ○ & ○ & ○ & ○ & \ & ○ & ○ & ○ \\ \hline
      node02 & ○ & ○ & ○ & ○ & ○ & \ & ○ & ○ \\ \hline
      node03 & ○ & ○ & ○ & ○ & ○ & ○ & \ & ○ \\ \hline
      node04 & ○ & ○ & ○ & ○ & ○ & ○ & ○ & \ \\ \hline
    \end{tabular}
  \end{center}
\end{table}

\subsection{Kubernetesクラスタの構築}

OpenVPNによる拠点間の接続を行った後,Kubernetesクラスタを構築した.
本研究の実装では,Kubeadmを使用した高可用性Kubernetesクラスタを構築するため,まず初めに複数マスターへのリクエストを振り分けるロードバランサーを設置した.
ロードバランサーの構築には,HA-Proxy 1.8.8を採用した.\\

\begin{lstlisting}
  frontend kubernetes
      bind *:6443
      option tcplog
      mode tcp
      default_backend kubernetes-backend

  frontend etcd
      bind *:2379
      option tcplog
      mode tcp
      default_backend etcd-backend

  backend kubernetes-backend
      mode tcp
      balance roundrobin
      option tcp-check
      server master01 192.168.10.101:6443 check
      server master02 192.168.10.102:6443 check
      server master02 192.168.10.103:6443 check

  backend etcd-backend
      mode tcp
      balance roundrobin
      server master01  192.168.10.101:2379 check
      server master02  192.168.10.102:2379 check
      server master03  192.168.10.103:2379 check
\end{lstlisting}

上記の設定で,ロードバランサーのポート6443番とポート2379番へのリクエスを三台のマスターノードへと振り分けている.

\begin{figure}[htbp]
  \begin{center}
    \includegraphics[width=0.4\textwidth]{./figures/haproxy.jpg}
    \caption{マスターノードとワーカーノードの関係性}
  \end{center}
\end{figure}

次に,マスターノードとワーカーノードを立ち上げるにあたり必要なパッケージをインストールした.
Kuberntesのランタイムとして使用するDockerに加え,クラスタ構築時に用いるkubeadmとkubelet, クラスタ操作時に必要なkubectlをaptによって取得した.
パッケージの用意が完了したのち,マスターノードからクラスタ構築作業を行った.
kubeadmではクラスタの初期化用にinitコマンドが用意されており,初めのマスターノードにて実行することでクラスタの基盤を作成可能である.
初期化に成功した場合,以下のようなテキストが出力される.\\

\begin{lstlisting}
  You can now join any number of control-plane nodes by copying certificate authorities
  and service account keys on each node and then running the following as root:

    kubeadm join 192.168.10.253:6443 --token { token } \
      --discovery-token-ca-cert-hash sha256:{ hash }} \
      --control-plane

  Then you can join any number of worker nodes by running the following on each as root:

  kubeadm join 192.168.10.253:6443 --token { token } \
      --discovery-token-ca-cert-hash sha256:{ hash }}
\end{lstlisting}

出力にある通り,与えられたコマンドを他のマスターノードとワーカーノードから実行することでクラスタへの参加が行える.
各サーバにて上記のコマンドを実行した結果,マスターノードからクラスタが構築できていることを確認できた.

\begin{lstlisting}
  $ kubectl get nodes
  NAME       STATUS     ROLES    AGE     VERSION
  master01   Ready      master   58d     v1.16.3
  master02   Ready      master   58d     v1.16.3
  master03   Ready      master   58d     v1.16.3
  node01     Ready      <none>   8d      v1.16.3
  node02     Ready      <none>   4d22h   v1.16.3
  node03     Ready      <none>   6d16h   v1.16.3
  node04     Ready      <none>   8d      v1.16.3
\end{lstlisting}

\section{システム全体}
\label{implementation:system}
本研究で構築した実装環境の図を以下に示す.

\begin{landscape}
  \begin{figure}[htbp]
    \begin{center}
      \includegraphics[width=\textwidth]{./figures/network-diagram.jpg}
      \caption{ネットワーク構成図}
    \end{center}
  \end{figure}
\end{landscape}

%%% Local Variables:
%%% mode: japanese-latex
%%% TeX-master: "../bthesis"
%%% End:
