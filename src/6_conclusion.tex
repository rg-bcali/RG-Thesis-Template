\chapter{結論}
\label{conclusion}

本章では,本研究のまとめと今後の課題を示す.

\section{まとめ}
\label{conclusion:conclusion}

本研究では,惑星規模の分散システムの試験を行うための試験環境の構築手法を提案した.
惑星規模の分散システムは,地理的に分散したコンピュータによって構成されるため,
試験においてはネットワーク上の通信の遅延を考慮する必要があることを~\ref{background}章で示した.
~\ref{issue}章では,既存の提案手法としてPlanetLab,クラウドサービスのリージョンの活用,BSafe.networkをあげたが,
それぞれに欠点があり未だ惑星規模の分散システムの試験環境の構築手法に課題が残されていることを指摘した.
加えて,惑星規模の分散システムの試験における以下四点の必要要件を明らかにした.
\begin{itemize}
  \item OSやCPU, Memoryといったサーバ環境を柔軟に変更可能であること
  \item 公の実稼働環境を想定したネットワークでの通信の遅延を考慮できること
  \item 異なる管理権限下にある各サーバに対し統合的管理が可能であり,各オペレータの手作業を軽減できること
  \item 地理的に分散した各サーバに対し,統合的な操作が可能であること
\end{itemize}
上記の必要要件を満たすため,本研究ではOpenVPNとKubernetesを組み合わせた惑星規模の分散システムのための構築手法を提案した.
試験環境を構成するサーバ間をOpenVPNオーバーレイネットワークによって繋げ,その上でKubernetesクラスタを構築することで,
地理的に分散したサーバを統合的に管理することができるのではないかと考えた.
~\ref{implementation}章では,提案手法の構築を行った.
ESXiを利用して仮想的にネットワーク環境を構築し,疎通性のないセグメント間でKubernetesクラスタを立ち上げた.
~\ref{evaluation}章では,~\ref{issue}章で明らかにした惑星規模の分散システムの試験環境における必要要件を本システムが満たせているか評価を行なった.
ネットワーク上の別セグメントに位置するサーバによるKubernetesクラスタが,各サーバに対し統括的な指示ができることより,
本システムが本研究の課題に対する必要要件を満たせていることを確認した.
本節の冒頭でも述べたように,惑星規模の分散システムではネットワークでの通信の遅延がシステムに影響を及ぼす.
そのため,通信の遅延を考慮した上で各コンピュータが協調動作できていることを試験しなければならず,惑星規模の分散システムのための試験環境を構築することは容易ではない.
本研究は,地理的に分散したコンピュータによって構成される分散システムの試験環境を構築手法を提案するものであり,
システムの堅牢性の向上に繋がったと考える.

\section{課題と展望}
\label{conclusion:issue}

本節では,本研究の課題と展望について述べる.
本研究の実装は,ESXiによって構築した仮想ネットワーク上で異なるセグメントを繋ぐOpenVPNオーバーレイネットワークを実装し,
さらにその上でKubernetesクラスタの構築を行なった.
実装はLAN内で行なったものであり,実際に地理的に分散したコンピュータを用いて実装が行えなかった点は課題として残された.
実用に向けた次の段階としては,本研究て提案したシステムを公のインターネット上で実装し再評価する必要があると考える.
Bitcoinの基盤技術であるブロックチェーンが登場したことによって,惑星規模の分散システムには今後より注目が集まると考える.
今後,惑星規模の分散システムの堅牢性をより一層向上させる必要があり,そのためには試験環境の提案手法の確立が大きな課題である.

%%% Local Variables:
%%% mode: japanese-latex
%%% TeX-master: "../thesis"
%%% End:
