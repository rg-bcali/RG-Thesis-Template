\chapter{結論}
\label{conclusion}

本章では,本研究のまとめと今後の課題を示す.

\section{本研究のまとめ}
\label{conclusion:conclusion}

本研究では,惑星規模の分散システムのテストを行うためのステージング環境における各地点のオペレータ間のコミュニケーションや手作業によって生じるオーバーヘッドを解決するため,
OpenVPNとKubernetesを利用したステージング環境の提案をした.
着目した課題に対する解決策として実際性,統合性,拡張性の三つの要件が求められると考えた.
第一に,OpenVPNを用いることでネットワーク上で論理的に離れたノード間での疎通性を獲得した.
これによって対象のノードを開発環境から実際のインターネット上に拡張することができ,惑星規模の分散システムのテストに必要である実際性を満たすことが出来たと考える.
第二に拡張性については,OpenVPNによるオーバーレイネットワーク上でKubernetesクラスタを構築することで,分散したコンピュータに対し統合的な操作を可能にすることで解決した.
第三に拡張性であるが,Kubernetesクラスタ上ではアプリケーションをコンテナ型仮想マシンとして動作させるため,容易にコンテナの追加や削除を行うことが可能である.
加えて,kubeadmによりクラスタへ新規にノードを追加することも可能であるため,ステージング環境を自由に拡張することが可能である.
よって拡張性も満たしていると考えられる.

\section{本研究の課題と展望}
\label{conclusion:issue}

本研究では,OpenVPNを用いたオーバーレイネットワーク上にKubernetesクラスタを構築した.
すべてのコンピュータはVPNサーバと接続し,Kubernetesクラスタ上での通信はすべてVPNサーバを通して行う.
本研究ではVPNサーバの負荷とそれに伴うKubernetesクラスタへの影響までを測定することができなかった.
実用に向けた次のステップとしては,VPNサーバの負荷とレイテンシについての詳細な実験をする必要性があると考えた.

%%% Local Variables:
%%% mode: japanese-latex
%%% TeX-master: "../thesis"
%%% End:
